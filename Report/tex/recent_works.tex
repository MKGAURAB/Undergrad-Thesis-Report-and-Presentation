\documentclass{standalone}
\usepackage{standalone}

\begin{document}
\section{Recent Works}

\subsection{KCMBT}

KCMBT is a tool for k-mer counting which implemented tree based algorithm. It speeds up based on multiple burst trees\cite{KMCBT}. It claimed that it has outperformed KMC 2 and presented the graph showed in Figure \ref{fig:KCMBT}. But the tools is not publicly accessible till now. As it is a conference paper, it seems to have some issues with reliability. 
\begin{figure}[ht]
	\centering
		\fbox{\includegraphics[scale=0.8]{./img/KCMBT}}
	\caption{Comparison among Jellyfish, KMC 2 and KCMBT.\cite{KCMBT}}
	\label{fig:KCMBT}
\end{figure}

\subsection{Benchmarking Paper}
A very recent assessment paper was published on April 4 of this year. Here, they experimented with almost all existing tools. As Jellyfish 1 could perform well till k = 31, they took this value of k. As a large value of k, they have taken k = 55. They have shown the comparison based on three important factors.
\begin{enumerate}
	\item Memory Usage (RAM)\ref{fig:Assess2}
	\item Run-time\ref{fig:Assess1}
	\item Disk I/O\ref{fig:Assess3}
\end{enumerate}

\begin{figure}[ht]
	\centering
		\fbox{\includegraphics[scale=0.7]{./img/AssessRunTime}}
	\caption{Comparison among existing K-mer counting tools with respect to Run-time.\cite{reviewAllInOne}}
	\label{fig:Assess1}
\end{figure}

\begin{figure}[ht]
	\centering
		\fbox{\includegraphics[scale=0.7]{./img/AssessMemory}}
	\caption{Comparison among existing K-mer counting tools with respect to Memory Usage.\cite{reviewAllInOne}}
	\label{fig:Assess2}
\end{figure}

\begin{figure}[ht]
	\centering
		\fbox{\includegraphics[scale=0.7]{./img/AssessDiskIO}}
	\caption{Comparison among existing K-mer counting tools with respect to Disk I/O.\cite{reviewAllInOne}}
	\label{fig:Assess3}
\end{figure}

There experimented result almost agrees with our analysis.

\end{document}