\documentclass{standalone}
\usepackage{standalone}

\begin{document}
\subsection{Suffix Array}
\subsubsection{Overview}
Suffix Array\cite{suffix_array} is a common data structure which keeps the starting index of sorted suffixes of a string. The definition of suffix is an important thing here. So, a suffix is a substring or consecutive characters from a string from any position to the end of the string. An example would make the thing more clear. So, lets take an example string,
$$S = ATTCGAGCATCAG$$
To keep the track of ending, an end mark like \$ should be appended.
$$S\$ = ATTCGAGCATCAG\$$$
The suffixes are shown in Figure \ref{fig:suffixUnsort}. After sorting lexicographically, the order of the suffixes becomes like Figure \ref{fig:suffixSort}. So, Suffix Array of S would be as in Figure \ref{fig:SA}.
\begin{figure}
	\centering
	\begin{BVerbatim}
	ATTCGAGCATCAG$ --  0
	 TTCGAGCATCAG$ --  1
	  TCGAGCATCAG$ --  2
	   CGAGCATCAG$ --  3
	    GAGCATCAG$ --  4
	     AGCATCAG$ --  5
	      GCATCAG$ --  6
	       CATCAG$ --  7
	        ATCAG$ --  8
	         TCAG$ --  9
	          CAG$ -- 10
	           AG$ -- 11
	            G$ -- 12
	             $ -- 13
	\end{BVerbatim}
	\caption{Suffixes of S\$ in ascending order of their starting index.}
	\label{fig:suffixUnsort}
\end{figure}

\begin{figure}
	\centering
\begin{BVerbatim}
$              -- 13
AG$            -- 11
AGCATCAG$      --  5
ATCAG$         --  8
ATTCGAGCATCAG$ --  0
CAG$           -- 10
CATCAG$        --  7
CGAGCATCAG$    --  3
G$             -- 12
GAGCATCAG$     --  4
GCATCAG$       --  6
TCAG$          --  9
TCGAGCATCAG$   --  2
TTCGAGCATCAG$  --  1
\end{BVerbatim}
	\caption{Suffixes of S\$ sorted in Lexicographical Order with their starting index.}
	\label{fig:suffixSort}
\end{figure}

\begin{figure}
	\centering
	\tikzstyle{block} = [rectangle, draw, line width=0.5mm,
	text centered]
	\tikzstyle{line} = [draw, -latex']
	\begin{tikzpicture}[auto]
	
	\node [rectangle,minimum width=2cm](lab) {$S\$ =$};
	\node [block, anchor=west] (Val1_1) at (lab.east) {A};
	\node [block, anchor=west] (Val1_2) at (Val1_1.east) {T};
	\node [block, anchor=west] (Val1_3) at (Val1_2.east) {T};
	\node [block, anchor=west] (Val1_4) at (Val1_3.east) {C};
	\node [block, anchor=west] (Val1_5) at (Val1_4.east) {G};
	\node [block, anchor=west] (Val1_6) at (Val1_5.east) {A};
	\node [block, anchor=west] (Val1_7) at (Val1_6.east) {G};
	\node [block, anchor=west] (Val1_8) at (Val1_7.east) {C};
	\node [block, anchor=west] (Val1_9) at (Val1_8.east) {A};
	\node [block, anchor=west] (Val1_10) at (Val1_9.east) {T};
	\node [block, anchor=west] (Val1_11) at (Val1_10.east) {C};
	\node [block, anchor=west] (Val1_12) at (Val1_11.east) {A};
	\node [block, anchor=west] (Val1_13) at (Val1_12.east) {G};
	\node [block, anchor=west] (Val1_14) at (Val1_13.east) {\$};
	%suffix array
	\node [rectangle,below of=lab,minimum width=2cm](lab2) {$SA(S) =$};
	\node [block, anchor=west] (Val2_1) at (lab2.east) {14};
	\node [block, anchor=west] (Val2_2) at (Val2_1.east) {12};
	\node [block, anchor=west] (Val2_3) at (Val2_2.east) {6};
	\node [block, anchor=west] (Val2_4) at (Val2_3.east) {9};
	\node [block, anchor=west] (Val2_5) at (Val2_4.east) {0};
	\node [block, anchor=west] (Val2_6) at (Val2_5.east) {11};
	\node [block, anchor=west] (Val2_7) at (Val2_6.east) {8};
	\node [block, anchor=west] (Val2_8) at (Val2_7.east) {4};
	\node [block, anchor=west] (Val2_9) at (Val2_8.east) {13};
	\node [block, anchor=west] (Val2_10) at (Val2_9.east) {5};
	\node [block, anchor=west] (Val2_11) at (Val2_10.east) {7};
	\node [block, anchor=west] (Val2_12) at (Val2_11.east) {10};
	\node [block, anchor=west] (Val2_13) at (Val2_12.east) {3};
	\node [block, anchor=west] (Val2_14) at (Val2_13.east) {2};
	
	
	\end{tikzpicture}
	\caption{Suffix Array of the string S, SA(S). } \label{fig:SA}
\end{figure}

\subsubsection{Building Suffix Array}
An approach of building suffix array could have the following steps.
\begin{itemize}
	\item Bulild a Suffix Tree as mentioned in the previous section.
	\item Do a Depth First Traversal storing indexes at leaves.
\end{itemize}
The memory complexity could be reduced following several approaches\cite{SA_building} with LCP array where the Suffix Tree should not be constructed. However,  traversing $n$ nodes need $O(n)$ time. So, it is the building time complexity of Suffix Array.

\subsubsection{Querying in Suffix Array}
A binary search algorithm\cite{binaru_search} could be used to query in the suffix array. Say we want to have the answer of following two questions where $P = AT$ would be used as query string.
\begin{itemize}
	\item Does P occurs in S?
	\item How many times does P occur in S?
\end{itemize}
To answer the first question, we would do a binary search. if any match found, then it occurs, otherwise not. Have a look at Figure \ref{fig:suffixSort} and see, it occurs. But here is steps in binary search.
\begin{itemize}
	\item left = 0, right = 13, mid = 6. SA[6] = 8.
	\item S[8] = 'A', S[8+1] = 'T' which matches with P[0] and P[1]. So, T is a substring of S.
\end{itemize}

However, to answer the second question, we should have two binary search. One would find left bound and other would find right bound. The answer is just the difference between them.
\subsubsection{Complexity Analysis}
The memory complexity of Suffix Array is $O(n)$ where $n$ is the length of the string. To index human genome it takes 12GB\cite{bowtie} where MUMmer\cite{MUMmer3} takes 47GB using Suffix Tree.

In case of searching, both of the answers in the querying section takes$O(m log_2(n))$ where, $n$ and $m$ are the length of the main string and query string respectively. Here, $m$ for comparison and $log_2(n)$ for binary search. But it could be reduced to $O(m +  log_2(n))$ in practice using LCP\cite{SA_LCP,SA_LCP1,SA_LCP2,SA_LCP3} and other techniques\cite{SA_LCP4,SA_LCP5,SA_LCP6}.
\end{document}